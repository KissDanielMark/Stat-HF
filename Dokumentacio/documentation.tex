\documentclass[11pt,a4paper,oneside]{report}   
\usepackage{listings}
\linespread{1.5}
\input{preamble}
\usepackage{graphicx}
\graphicspath{ {./images/} }

\title{\Huge{MSc - Statisztika}\\Házi feladat}
\author{\huge{Kiss Dániel Márk}}
\date{2023}

\begin{document}

\maketitle
\newpage
\tableofcontents
\pagebreak

\chapter{1. feladat}

Az elmúlt évek kutatásai arra irányultak, hogy felmérjék a mosolygós emojik használatának hatását a digitális kommunikációban
és a felhasználók boldogságszintjére. Az alábbi adatokkat gyűjtötték össze: bead11.1.csv.

Lineáris regressziós modellt szeretnénk felírni, melyben az eredményváltozó a boldogságszint, míg a magyarázó változók az üzenet hossza és
a mosolygós emojik száma.
\section{a) feladat}
Beccsüld meg és értelmezd a lineáris regresszió paramétereit, teszteld le, szignifikánsak-e a magyarázó változók!(5\%-os szignifikanciaszinten)

Megoodlás:


\section{b) feladat}
Határozd meg és értelmezd a többszörös determináiós együtthatót!

Megoodlás:

\section{c) feladat}
Teszteld a regressziós modell megbízhatóságát 5\%-os szignifikanciaszinten!

Megoodlás:

\section{d) feladat}
Adj intervallumbecslést 95\%-os megbizhatósággal paraméterekre!

Megoldás:

\section{e) feladat}
Készíts előrejelzést az új üzenetek boldogságszintjére, ha az üzenet hossza 130 karakter,  és a mosolygós emojik száma 3. Illetve adj ugyanerre 95\%-os megbízhatóságú intervallumbecslést is.

Megoodlás:


\chapter{2.feladat}
A következő kutatás arra irányult, hogy mérje a mosolygós emojik használatának hatását a kommunikációban különböző csoportokban. Az alábbi adatokat gyűjtötték  össze: bead11.1.csv.

\section{a) feladat}
Teszteld le, hogy van-e szignifikáns különbség a mosolygós emojik használatának gyakoriságában a különböző csoportokban (E = 0,05 szignifikanciaszinten)!

Megoldás:

\chapter{3.feladat}
A bead11.3.csv file egy felmérés adatait mutatja a mosolygós emojik használatának változásáról az elmúlt években egy adott online fórumon.

\section{a) feladat}
Készíts idősor diagramot az adatok alapján, majd számold ki a tapasztalati autokorrelációs és parciális autokorrelációs függvényeket.

Megoldás:
\section{b) feladat}
Az adatok transzformációjával és a trend, valamint a szezonális komponensek kiszűrésével kísérletezve illessz különböző idősor modelleket. Teszteld az illeszkedést.

Megoodlás:

\section{c) feladat}
Készíts előrejelzést a következő hónapokra várható mosolygós emojik használatára.

Megoldás:


\end{document}
