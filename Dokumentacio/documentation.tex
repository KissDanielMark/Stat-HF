\documentclass[11pt,a4paper,oneside]{report}   
\usepackage{listings}
\linespread{1.0}
\input{preamble}
\usepackage{graphicx}
\graphicspath{ {./images/} }

\title{\Huge{MSc - Statisztika}\\Házi feladat}
\author{\huge{Kiss Dániel Márk}}
\date{2023}

\begin{document}

\maketitle
\newpage
\tableofcontents
\pagebreak

\chapter{1. feladat}

Az elmúlt évek kutatásai arra irányultak, hogy felmérjék a mosolygós emojik használatának hatását a digitális kommunikációban
és a felhasználók boldogságszintjére. Az alábbi adatokkat gyűjtötték össze: bead11.1.csv.

Lineáris regressziós modellt szeretnénk felírni, melyben az eredményváltozó a boldogságszint, míg a magyarázó változók az üzenet hossza és
a mosolygós emojik száma.
\section{a) feladat}
Beccsüld meg és értelmezd a lineáris regresszió paramétereit, teszteld le, szignifikánsak-e a magyarázó változók!(5\%-os szignifikanciaszinten)

Megodlás:
A bead11.1.csv fájl négy oszlopot tartalmaz: "Sorszám," "Üzenet hossza," "Mosolygós emojik száma," és "Boldogságszint."
A lineáris regresszió célja az, hogy a függő változót (pl. Boldogságszint) lineáris kapcsolatban álló magyarázó változókkal (pl. Üzenet hossza, Mosolygós emojik száma) modellezze.

A lineáris regresszió modellje általánosan a következő alakú:
Y=B0+B1X1+B2X2+E.

Kimenet értelmezése:
\begin{figure}[!ht]
  \begin{center}
    \includegraphics[scale=0.4]{1.1.png}
    \caption{Feladat 1/a kimenet}
    \label{fig:TexnicCenter}
  \end{center}
\end{figure}
Az alábbi kimenet a Figure 1.1-en látható. Az R-négyzet érték azt mutatja, hogy a modell mennyire magyarázza a függő változó (Boldogságszint) változását. Az 0.870 érték azt jelenti, hogy a modell 87\%-ban magyarázza a változást.
P \> abs(t) (szignifikanciaszint): Az egyes együtthatók (const, Üzenet hossza, Mosolygós emojik száma) szignifikanciaszintje. Az értékek alattuk a p-értékeket jelentik. Azok az együtthatók, amelyek p-értéke kevesebb, mint 0.05, szignifikánsak a 0.05 szignifikanciaszinten. Ebben a modellben mind a const (konstans), mind a Mosolygós emojik száma szignifikáns, mivel a p-értékük kisebb, mint 0.05.
Összességében ez azt jelenti, hogy a modell jól teljesít a magyarázatokban, és mind a konstans, mind a Mosolygós emojik szám változói szignifikánsan kapcsolódnak a Boldogságszint változóhoz.
\section{b) feladat}
Határozd meg és értelmezd a többszörös determináiós együtthatót!

Megodlás:

A többszörös determináiós együttható (R-négyzet) azt mutatja, hogy a modell mennyire magyarázza a függő változó (Boldogságszint) változását. Az 0.870 érték azt jelenti, hogy a modell 87\%-ban magyarázza a változást.
Az R érték 0 és 1 közötti értéket vehet fel. Minél közelebb van az 1-hez, annál jobban magyarázza a modell a függő változó (Boldogságszint) változását.
Az R-négyzet mellett fontos megjegyezni az "Adj. R-squared" értéket is (itt 0.865), amely korrigálja az R-négyzetet a magyarázó változók számára. Ez különösen fontos, ha több magyarázó változó van a modellben, mivel az R-négyzet hajlamos növekedni a változók számával anélkül, hogy ténylegesen javítaná a modell illeszkedését.

\section{c) feladat}
Teszteld a regressziós modell megbízhatóságát 5\%-os szignifikanciaszinten!

Megoldás:
A nullhipotézis az, hogy a modell nem szignifikáns azaz nincs összefüggés az emojik és a boldogságszint között. A nullhipotézis elutasításához a p-értéknek kisebbnek kell lennie, mint a szignifikanciaszint (5\%).
Az adott kimenetben a F-statistic értéke 157.5, és a hozzá tartozó p-érték a "Prob (F-statistic)" oszlopban található (1.46e-21). Ez az érték rendkívül kicsi, sok nagyságrenddel kisebb, mint 0.05 (5\%-os szignifikanciaszint), így elvetjük a nullhipotézist (azaz elfogadjuk a modell szignifikanciáját). Ez azt jelenti, hogy a modell összességében szignifikánsan jól illeszkedik adatainkhoz.


\section{d) feladat}
Adj intervallumbecslést 95\%-os megbizhatósággal paraméterekre!

Megoldás:
Az intervallumok azt mutatják, hogy a konstans érték (const) becslési intervalluma 4.493509 és 6.016082 között van,
az Üzenet hossza becslési intervalluma -0.001479 és 0.016810 között van,
míg a Mosolygós emojik száma becslési intervalluma 0.166219 és 0.447623 között van.
Ezek az intervallumok segíthetnek abban, hogy becsüljük a paraméterek értékeinek megbízhatóságát és azt, mennyire pontosak a becslések.

\begin{tabular}{|c|c|c|}
  \hline
                         & 0         & 1        \\
  \hline
  const                  & 4.493509  & 6.016082 \\
  \hline
  Üzenet hossza          & -0.001479 & 0.016810 \\
  \hline
  Mosolygós emojik száma & 0.166219  & 0.447623 \\
  \hline
\end{tabular}




\section{e) feladat}
Készíts előrejelzést az új üzenetek boldogságszintjére, ha az üzenet hossza 130 karakter,  és a mosolygós emojik száma 3. Illetve adj ugyanerre 95\%-os megbízhatóságú intervallumbecslést is.

Megoldás:
Előrejelzés: 7.172034019616269
95\%-os megbízhatóságú intervallum: 7.09557490761663 - 7.248493131615907

Ez azt jelenti, hogy az új üzenetek boldogságszintje várhatóan körülbelül 7.2 lesz,
és a 95\%-os megbízhatóságú intervallum körülbelül 7.1 és 7.2 között lesz.



\chapter{2.feladat}
A következő kutatás arra irányult, hogy mérje a mosolygós emojik használatának hatását a kommunikációban különböző csoportokban. Az alábbi adatokat gyűjtötték  össze: bead11.1.csv.

\section{a) feladat}
Teszteld le, hogy van-e szignifikáns különbség a mosolygós emojik használatának gyakoriságában a különböző csoportokban (E = 0,05 szignifikanciaszinten)!

Megoldás:
Ahhoz, hogy leellenőrizzük, van-e szignifikáns különbség a mosolygós emojik használatának gyakoriságában a különböző csoportokban, statisztikai tesztet kell alkalmazni.
A leggyakrabban használt teszt a kétváltozós t-próba. Azonban, mivel itt több csoportról van szó, egy análízis varianciát (ANOVA) is érdemes megfontolni.

Kimenet értelmezése:
Statisztika: 25.558435652569365, p-érték: 5.339953301217143e-14
Van szignifikáns különbség a csoportok között a mosolygós emojik használatában.
A kapott statisztika és p-érték alapján, azt mondhatjuk, hogy van szignifikáns különbség a csoportok között a mosolygós emojik használatában. A p-érték rendkívül kicsi, jóval kisebb az elfogadható 0.05 szignifikanciaszintnél. Ez azt sugallja, hogy a csoportok közötti különbség valószínűleg nem véletlen, és a mosolygós emojik használata szignifikánsan eltérő a csoportokban.

\chapter{3.feladat}
A bead11.3.csv file egy felmérés adatait mutatja a mosolygós emojik használatának változásáról az elmúlt években egy adott online fórumon.

\section{a) feladat}
Készíts idősor diagramot az adatok alapján, majd számold ki a tapasztalati autokorrelációs és parciális autokorrelációs függvényeket.

Megoldás:

\begin{figure}[!ht]
  \begin{center}
    \includegraphics[scale=0.4]{Figure_1.png}
    \caption{Feladat 3/a kimenet}
    \label{fig:TexnicCenter}
  \end{center}
\end{figure}

\begin{figure}[!ht]
  \begin{center}
    \includegraphics[scale=0.4]{Figure_2.png}
    \caption{Feladat 3/a kimenet}
    \label{fig:TexnicCenter}
  \end{center}
\end{figure}

\begin{figure}[!ht]
  \begin{center}
    \includegraphics[scale=0.4]{Figure_3.png}
    \caption{Feladat 3/a kimenet}
    \label{fig:TexnicCenter}
  \end{center}
\end{figure}
\section{b) feladat}
Az adatok transzformációjával és a trend, valamint a szezonális komponensek kiszűrésével kísérletezve illessz különböző idősor modelleket. Teszteld az illeszkedést.

Megoodlás:

\begin{figure}[!ht]
  \begin{center}
    \includegraphics[scale=0.4]{Figure_4.png}
    \caption{Feladat 3/b kimenet}
    \label{fig:TexnicCenter}
  \end{center}
\end{figure}

\begin{figure}[!ht]
  \begin{center}
    \includegraphics[scale=0.4]{Figure_5.png}
    \caption{Feladat 3/b kimenet}
    \label{fig:TexnicCenter}
  \end{center}
\end{figure}

\section{c) feladat}
Készíts előrejelzést a következő hónapokra várható mosolygós emojik használatára.

Megoldás:

\begin{figure}[!ht]
  \begin{center}
    \includegraphics[scale=0.4]{Figure_6.png}
    \caption{Feladat 3/c kimenet}
    \label{fig:TexnicCenter}
  \end{center}
\end{figure}


\end{document}
